\documentclass[11pt, a4paper]{article}

\usepackage{ucs}
\usepackage{amsmath}
\usepackage[utf8x]{inputenc}   
\usepackage[russian]{babel}
\usepackage[a4paper, top=2cm, bottom=2cm, left=3cm, right=3cm, marginparwidth=1.75cm]{geometry} 

\title{Лабораторная работа №2}
\author{Халина Ирина}


\makeatletter
\renewcommand{\@maketitle}{
	\begin{center}
		\begin{huge}
			\par\textbf{\@title}
		\end{huge}
		\vspace{0.3cm}
		\begin{large}
		\par\textit{\@author}
		\end{large}
	\end{center}
}
\makeatother


\begin{document}
	\par\maketitle
	
	\begin{center}
	\subsection*{Вариант 1}
	\end{center}
	
	\subsection*{Задание 1}
	
	\par
	a) Уравнение в полных дифференциалах. Метод:
	\begin{equation*}
		\int_{x_0}^{x}\left(2xy^2+3x^2+\frac{1}{x^2}+\frac{3x^2}{y^3}\right)dx+\int_{y_0}^{y}\left(2x_0y+3y^2+\frac{1}{y^2}-\frac{3x_0^3}{y^3}\right)dy=u(x,y).
	\end{equation*}
	
	\noindent
	б) Уравнение с разделяющимися переменными. Метод:
	\begin{equation*}
		\int \frac{xdx}{1+x^2}=-\int \frac{dy}{y(1+y^2)}.
	\end{equation*}
	
	\noindent
	в) Однородное уравнение. Метод: замена $\frac{y}{x}=u$.
	
	\noindent
	г) Уравнение, линейное по $y$. Метод: метод Лагранжа.
	
	\noindent
	д) Уравнение Бернулли. Метод: замена переменных $u=y^3$.
	
	\noindent
	е) Уравнение Риккати. Метод: не решается в общем случае.
	
	\noindent
	ж) Уравнение, линейное по $x$. Метод: метод Лагранжа.

	
	\subsection*{Задание 2}
	
	\par
	\begin{equation*}
    	(x^3+x^3\ln{x}+2y)dx+(3y^2x^3-x)dy=0,
	\end{equation*}
	\begin{equation*}
    	\Psi(y): \quad \frac{Q'_x-P'_y}{P}=\frac{9x^2y^2-3}{x^3+x^3\ln(x)+2y}=f(x,y) \Rightarrow \Psi\neq\Psi(y),
	\end{equation*}
	\begin{equation*}
    	\Psi(x): \quad \frac{P'_y-Q'_x}{Q}=\frac{3-9x^2y^2}{3x^3y^2-x}=\frac{3(1-3x^2y^2)}{-x(1-3x^2y^2)}=-\frac{3}{x}\Rightarrow \Psi=\Psi(x),
	\end{equation*}
	\begin{equation*}
    	\mu(x)=e^{\int\Psi(x)dx}=e^{-\int\frac{3}{x}dx}=e^{-3\ln x}=x^{-3},
	\end{equation*}
	\begin{equation}
    	\left(1+\ln(x)+\frac{2y}{x^3}\right)dx+\left(3y^2+-\frac{1}{x^2}\right)dy=0 \label{pd},\quad x \neq 0,
	\end{equation}
	\begin{equation*}
    	\frac{\partial P_1}{\partial y}=\frac{\partial Q_1}{\partial x}=\frac{2}{x^3},
	\end{equation*}
	(\ref{pd}) - уравнение в полнных дифференциалах.
	\begin{equation*}
    	\frac{\partial u}{\partial x} = 1+\ln x+\frac{2y}{x^3},
	\end{equation*}
	\begin{equation*}
    	u(x,y)= \int \left(1+\ln x+\frac{2y}{x^3}\right)dx,
	\end{equation*}
	\begin{equation*}
    	u(x,y)= x\ln x-\frac{y}{x^2}+C(y),
	\end{equation*}
	\begin{equation*}
    	\frac{\partial u}{\partial y} = -\frac{1}{x^2}+C'(y),
	\end{equation*}
	\begin{equation*}
    	C'(y)=3y^2,
	\end{equation*}
	\begin{equation*}
    	C(y)=y^3,
	\end{equation*}
	\begin{equation*}
   		C=x\ln x-\frac{y}{x^2}+y^3,
	\end{equation*}
	\begin{equation*}
    	x=0 \Rightarrow  u'_xdx+u'_ydy=0 \Rightarrow x=0\text{ - решение.}  
	\end{equation*}
	Ответ: $C=x\ln x-\frac{y}{x^2}+y^3$, $x=0$.
	
	
	\subsection*{Задание 3}
	
	\par
	\begin{equation*}
    	y(x+\ln y)+(x-\ln y)y'=0 \quad \ln y = \eta \quad d\eta=\frac{dy}{y},
	\end{equation*}
	\begin{equation*}
    	\frac{dy}{d\eta}(x+\eta)+(x-\eta)\frac{dy}{dx}=0 \quad |:dy,
	\end{equation*}
	\begin{equation*}
    	(x+\eta)dx+(x-\eta)d\eta=0,
	\end{equation*}
	\begin{equation*}
    	\frac{\partial P}{\partial \eta}=\frac{\partial Q}{\partial x}=1 \Rightarrow \text{уравнение в полных дифференциалах}.
	\end{equation*}
	Метод решения:
	\begin{equation*}
    	\int_{x_0}^{x}(x+\eta)dx+\int_{\eta_0}^{\eta}(x-\eta)d\eta=u(x,y),
	\end{equation*}
	\begin{equation*}
    	\frac{x^2}{2}+x\eta-\frac{\eta^2}{2}=C,
	\end{equation*}
	\begin{equation*}
    	\frac{x^2}{2}+x\ln y-\frac{\ln^2{y}}{2}=C.
	\end{equation*}
	Ответ:$\frac{x^2}{2}+x\ln y-\frac{\ln^2{y}}{2}=C$.
	
	
	\subsection*{Задание 4}
	
	\par
	\begin{equation*}
    	dy=(y-2)^{\frac{2}{3}}dx,\quad y|_{x=1}=2,
	\end{equation*}
	\begin{equation*}
    	d(y-2)(y-2)^{-\frac{2}{3}}=dx, \quad y-2=t,
	\end{equation*}
	\begin{equation*}
    	\int t^{-\frac{2}{3}}dt=\int dx,
	\end{equation*}
	\begin{equation*}
    	3t^{\frac{1}{3}}=x+C,
	\end{equation*}
	\begin{equation*}
    	3(y-2)^{\frac{1}{3}}=x+C,
	\end{equation*}
	\begin{equation*}
    	0=1+C \Rightarrow C=-1,
	\end{equation*}
	\begin{equation*}
    	3(y-2)^{\frac{1}{3}}=x-1.
	\end{equation*}
	Ответ: $3(y-2)^{\frac{1}{3}}=x-1$.
	
	
	\begin{center}
	\subsection*{Вариант 2}
	\end{center}
	
	\subsection*{Задание 1}
	
	\par
	a) Уравнение с разделяющимися переменными. Метод:
	\begin{equation*}
		\frac{x}{1 + x^2}dx - \frac{1}{y(1+y^2)}dy = 0.
	\end{equation*}
	
	\noindent
	б) Однородное уравнение. Метод: замена $\frac{y}{x}=u$.
	
	\noindent
	в) Однородное уравнение Метод: замена $\frac{y}{x}=u$.
	
	\noindent
	г) Уравнение, линейное по $y$. Метод: метод Лагранжа.
	
	\noindent
	д) Уравнение Бернулли. Метод: замена переменных $u=y^3$.
	
	\noindent
	е) Уравнение Риккати. Метод: 
	\begin{equation*}
		y' + \frac{1}{x}y = xy^2 + \frac{1}{x}.
	\end{equation*}
	
	\noindent
	ж) Уравнение, линейное по $x$. Метод: метод Лагранжа.

	
	\subsection*{Задание 2}
	
	\par
	\begin{equation*}
		y^2(x - y)dx + (1 - xy^2)dy = 0,
	\end{equation*}
	\begin{equation*}
    	\int{y^2(x - y)dx} + \int{(1 - xy^2)dy} = C,
	\end{equation*}
	\begin{equation*}
    	x^2 - 2xy - \frac{2}{y} = C,
	\end{equation*}
	\begin{equation*}
    	y = 0,
	\end{equation*}
	\begin{equation*}
		\mu = \mu(y): \frac{Q'_x - P'_y}{P} = \frac{-y^2 - (2xy - 3y^2)}{y^2(x - y)},
	\end{equation*}
	\begin{equation*}
		\mu = \mu(x): \frac{P'_y - Q'_x}{Q} = \frac{2xy - 3y^2 - y^2}{1 - xy^2} = \frac{2y(x - 2y)}{1 - xy^2},
	\end{equation*}
	\begin{equation*}
    	\mu' = \frac{2}{y^2}\mu,
	\end{equation*}
	\begin{equation*}
    	\frac{d\mu}{\mu} = - \frac{2dy}{y^2},
	\end{equation*}
	\begin{equation*}
   		\ln{\mu} = \ln{y^{-2}},
	\end{equation*}
	\begin{equation*}
    	\mu = y^{-2}.
	\end{equation*}
	Ответ: $ x^2 - 2xy - \frac{2}{y} = C, y = 0$.
	
	
	\subsection*{Задание 3}
	
	\par
	\begin{equation*}
    	y' = x + e^{x + 2y},
	\end{equation*}
	\begin{equation*}
   		\eta = e^{-2y}, 
	\end{equation*}
	\begin{equation*}
    	d\eta = -2e^{-2y}dy,
	\end{equation*}
	\begin{equation*}
    	\frac{dy}{dx} = x + e^{x + 2y},
	\end{equation*}
	\begin{equation*}
    	e^{-2y}dy = (xe^{-2y} + e^x)dx,
	\end{equation*}
	\begin{equation*}
    	-\frac{1}{2}d\eta = (x\eta + e^x)dx,
	\end{equation*}
	\begin{equation*}
    	\eta' + 2x\eta = -2e^x \text{ - линейное по } \eta.
	\end{equation*}
	Ответ: $\eta' + 2x\eta + ae^x = 0$.
	
	
	\subsection*{Задание 4}
	\par
	\begin{equation*}
    	dy = x\sqrt{y} dx,\quad y|_{x=1} = 0,
	\end{equation*}
	\begin{equation*}
    	xdx = \frac{dy}{\sqrt{y}},
	\end{equation*}
	\begin{equation*}
    	\frac{x^2}{2} = 2\sqrt{y} + C,
	\end{equation*}
	\begin{equation*}
   		C = \frac{1}{2},
	\end{equation*}
	\begin{equation*}
    	x^2 - 4\sqrt{y} = 1.
	\end{equation*}
	Ответ: $x^2 - 4\sqrt{y} = 1$.
\end{document}