\documentclass{article}
\usepackage[russian]{babel}
\usepackage{mathtext}
\usepackage{amsmath}

\title{ Лабораторная работа 3}
\author{Лиходиевский Аркадий}
\date{2022}

\begin{document}

\maketitle

\section*{Вариант 2}
\section*{1) Проинтегрировать ур-ия Лагранжа и Клеро}
\subsection*{a) $ 2xy' - y = cosy'$}

\begin{equation*}
    y = 2xy' - cosy' \text{ - уравнение Лагранжа}
\end{equation*}

\begin{equation*}
    \text{ Замена } y' = p \Rightarrow dy = pdx, dy = 2pdx + 2xdp + \sin{(p)}dp
\end{equation*}

\begin{equation*}
    2pdx + (2x + \sin{(p)})dp = pdx, -pdx = (2x + \sin{(p)})dp
\end{equation*}

\begin{equation*}
    -px' = 2x + \sin{(p)} , x' + \frac{2}{p} = -\frac{\sin{(p)}}{p} \text{ - линейное, } p \ne 0
\end{equation*}

Решаем методом Лагранжа:

\begin{equation*}
    x' + \frac{2}{p} = 0, \frac{dx}{x} = -\frac{2}{p}dp, \ln{x} = \ln{p^{-2}}
\end{equation*}

\begin{equation*}
    x = cp^{-2}
\end{equation*}

\begin{equation*}
    c = c(x):    
\end{equation*}

\begin{equation*}
  x' = c'p^{-2} - 2cp^{-3}
\end{equation*}

\begin{equation*}
    c'p^{-2} - 2cp^{-3} + 2cp^{-2} = -\frac{\sin{(p)}}{p}, c' = -p\sin{(p)}
\end{equation*}

\begin{equation*}
    c = \int{p\sin{(p)}dp} = p\cos{(p)} - \sin{(p)} + c_1
\end{equation*}

\begin{equation*}
    x = \frac{1}{p^2}(p\cos{(p)} - \sin{(p)} + c_1) = \frac{1}{p}\cos{(p)} - \frac{1}{p^2}\sin{(p)} + \frac{c_1}{p^2}
\end{equation*}

\begin{equation*}
    p = 0: y' = 0 \Rightarrow y = const \Rightarrow y = -1
\end{equation*}
\\
Ответ:
$
    \begin{cases}
    x = \frac{\cos{(p)}}{p} + \frac{c_1 - \sin{(p)}}{p^2} \\
    y = \cos{(p)} + 2\frac{c_1 - \sin{(p)}}{p} \\
    \end{cases} , y = -1
$
\subsection*{б) $y = xy' - (2 + (y')^2)$ - уравнение Клеро} 

 \text{ Замена }  y' = p 

\begin{equation*}
   \Rightarrow  y = px - (2 + p^2)
\end{equation*}

\begin{equation*}
    dy = xdp + pdx - 2pdp,  dy = pdx
\end{equation*}

\begin{equation*}
    xdp +pdx - 2pdp = pdx \Rightarrow xpd - 2pdp = 0 \Rightarrow (x - 2p) dp = 0
\end{equation*}

\begin{equation*}
    dp = 0 \Rightarrow  p = const
\end{equation*}

\begin{equation*}
    y' = p \Rightarrow y = c_1x + c_2
\end{equation*}

\begin{equation*}
    x = 2p
\end{equation*}

\begin{equation*}
    y' = \frac{x}{2} \Rightarrow y = \frac{x^2}{4} + C
\end{equation*}

Ответ: 
\begin{equation*}
    y = \frac{x^2}{4} + C,
    y = c_1x+c_2,  
\end{equation*}

\section*{2) Проинтегрировать уравнение}
\begin{equation*}
    2yy'' + (y')^2 = 0
\end{equation*}

Уравнение не содержащее явно x:
\begin{equation*}
    F(y, y', y'') = 0
\end{equation*}

Замена:
\begin{equation*}
    y' = p(y) 
\end{equation*}
\begin{equation*}
    y'' = p'y' = p'p
\end{equation*}
\begin{equation*}
    2ypp' + p^2 = 0, p' + \frac{p}{2y} = 0, p \ne 0 , y \ne 0
\end{equation*}
\begin{equation*}
    \frac{dp}{p} = -\frac{dy}{2y} , \ln{(p)} = \ln{(cy^{-\frac{1}{2}})}
\end{equation*}

\begin{equation*}
    p = c y^{-\frac{1}{2}}, y' = cy^{-\frac{1}{2}}, \frac{dy}{dx} = \frac{c}{y^{\frac{1}{2}}}
\end{equation*}

\begin{equation*}
    y^{\frac{1}{2}}dy = cdx
\end{equation*}

\begin{equation*}
    \frac{2}{3}y^{\frac{3}{2}} = c_1x + c_2
\end{equation*}

Особое решение:
p = 0, y' = 0 \Rightarrow y = const
Ответ: $ y = const$

\section*{3) Понизить порядок ур-ия и указать тип и метод интегрирования полученного ур-ия}
\begin{equation*}
    (y'')^2 + y' = xy''
\end{equation*}
Замена:
\begin{equation*}
    y' = p \Rightarrow y'' = p'
\end{equation*}
\begin{equation*}
    (p')^2 + p = xp
\end{equation*}

\begin{equation*}
   p = xp' - (p')^2 \text{ - уравнение Клеро}
\end{equation*}

Выполним замену: p' = t и решаем как в номере 1)

\section*{4) Проинтегрировать ур-ие в точных призводных}
\begin{equation*}
    yy'' + (y')^2 = 1
\end{equation*}
Подобрали первообразные слева и справа
\begin{equation*}
    yy' = x + c
\end{equation*}
\begin{equation*}
    y' = \frac{x + c}{y}, \frac{dy}{dx} = \frac{x + c}{y}
\end{equation*}
\begin{equation*}
    ydy = (x + c) dx
\end{equation*}
\begin{equation*}
    \frac{y^2}{2} = \frac{x^2}{2} + c_1x + c_2
\end{equation*}


Ответ:  $y^2 = x^2 + c_1x + c_2



\end{document}