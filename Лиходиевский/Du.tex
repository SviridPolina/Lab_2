\documentclass{article}
\usepackage[utf8]{inputenc}
\usepackage{amsmath}
\usepackage{setspace}
\usepackage{amssymb}
\usepackage{letltxmacro}
\usepackage[12pt]{extsizes}

\title{Differential equations, 1 lab}

\author{Likhodievskiy Arkady}
\date{2022}

\begin{document}
\maketitle
\newpage

169 - 171 Having found a particular solution by selection, bring this Riccati equation to the Bernoulli equations and solve them

\begin{flushleft}
{\bf 169 } $xy'-(2x+1)y+y^2=-x^2$

\end{flushleft}


\begin{left} 
\begin{spacing}{1.5}
$xy'-(2x+1)y+y^2=-x^2$

$y_0 = x - $privat desision
\end{spacing}
\end{left}
Examination:
\begin{left} 
\begin{spacing}{1.5}
$x -(2x+1)x+x^2=-x^2$

$-x^2=-x^2$
\end{spacing}
\end{left}
Replacement:
\begin{left} 
\begin{spacing}{2.5}
$y=U+y_0 = U+x$

$\displaystyle y'=U'+1=\frac{2x+1}{x}y-\frac{y^2}{x}-x$

$\displaystyle U'=-1+\frac{2x+1}{x}(U+x)-\frac{(y+x^2)}{x}-x$

$\displaystyle U'=-1++2U+2x+\frac{U}{x}+1-\frac{U^2}{x}-x-2U-x$

\fbox{
$\displaystyle U'=-\frac{U^2}{x}+\frac{U}{x}-$ Bernoulli equation
}
\end{spacing}
\end{left}
Replacement for Bernoulli's equation:
\begin{left}
\begin{spacing}{2.2}
$m=2; $ $\displaystyle \upsilon=U^{1-m}=\frac{1}{U}$

$\displaystyle U=\frac{1}{\upsilon}$

$\displaystyle U'=-\frac{\upsilon'}{\upsilon^2}$

$\displaystyle -\frac{\upsilon'}{\upsilon^2}=-\frac{1}{\upsilon^2 x}+\frac{1}{\upsilon x}$

$\displaystyle -\upsilon'=-\frac{1}{x}+\frac{\upsilon}{x}$

$\displaystyle -\upsilon'-\frac{\upsilon}{x}=0$

$\displaystyle -\upsilon'=\frac{\upsilon}{x}$

$\displaystyle -\frac{d\upsilon}{\upsilon}=\frac{dx}{x}$

$\ln(\upsilon)=-\ln(Cx)$

$\displaystyle \upsilon=C\frac{1}{x}$

$\displaystyle \upsilon'=C'\frac{1}{x}-\frac{1}{x^2}C$

$\displaystyle C'\frac{1}{x}-\frac{1}{x^2}C+C\frac{1}{x^2}=\frac{1}{x}$

$C'=1 \Rightarrow C= \widetilde{C}+x$

$\displaystyle \upsilon=\frac{\widetilde{C}}{x}+1$

$\displaystyle U=\frac{x}{\widetilde{C}+x}$

\fbox{
$\displaystyle y=x+\frac{x}{\widetilde{C}+x}$
}
\end{spacing}
\end{left}

\begin{flushleft}
{\bf 170 } $y'-2xy+y^2=5-x^2$


\end{flushleft}
privat desision:
\begin{left}
\begin{spacing}{1.5}
$y=ax+b$

$a-2ax^2-2xb+a^2x^2+b^2+2axb=5-x^2$
\begin{equation*}
 \begin{cases}
   a^2-2a=-1 \Rightarrow (a-1)^2=0 \Rightarrow a=1
   \\
   -2b+2ab=0
   \\
   a+b^2=5 \Rightarrow b^2=5-a=5-1=4 \Rightarrow b=\pm2
 \end{cases}
\end{equation*}

$y_0=x+2 - $privat desision
\end{spacing}
\end{left}
Replacement:
\begin{left} 
\begin{spacing}{1.8}
$y=U+x+2$

$y'=U'+1=5-x^2+2xy-y^2$

$y'=5-x^2+2xU+2x^2+4x-U^2-x^2-4-2Ux-4x-4U$

\fbox{
$U'+1=-U^2-4U-$ Bernoulli equation
}
\end{spacing}
\end{left}

Replacement for Bernoulli's equation:
\begin{left}
\begin{spacing}{2.2}
$m=2; $ $\displaystyle \upsilon=U^{1-m}=\frac{1}{U}$

$\displaystyle U=\frac{1}{\upsilon}$

$\displaystyle U'=-\frac{\upsilon'}{\upsilon^2}$

$\displaystyle -\frac{\upsilon'}{\upsilon^2}=-\frac{1}{\upsilon^2}-\frac{4}{\upsilon}$

$-\upsilon'=-1-4\upsilon$

$-\upsilon'+4\upsilon=0$

$\displaystyle \frac{d\upsilon}{\upsilon}=4dx$

$ln(\upsilon)=4ln(Ce^x)$

$\upsilon=Ce^{4x}$

$\upsilon'=C'e^{4x}+4Ce^{4x}$

$-C'e^{4x}-4Ce^{4x}=-1-4Ce^{4x}$

$\displaystyle C'=e^{-4x} \Rightarrow C=-\frac{1}{4}e^{-4x}+\widetilde{C}$

$\displaystyle \upsilon=-\frac{1}{4}+\widetilde{C}e^{4x}$

$\displaystyle U=\frac{1}{-\frac{1}{4}+\widetilde{C}e^{4x}}$

\fbox{
$\displaystyle y=x+2+\frac{4}{\widetilde{C}e^{4x}-1}$ 
}
\end{spacing}
\end{left}

\begin{flushleft}
{\bf 171 } $y'+2ye^x-y^2=e^{2x}+e^x$

\end{flushleft}
privat desision:
\begin{left}
\begin{spacing}{1.5}
$y=e^x+a$

$e^x+2e^{2x}+2e^xa-e^{2x}-a^2-2e^xa=e^{2x}+e^x$

$-a^2=0 \Rightarrow a=0$

$y_0=e^x - $privat desision
\end{spacing}
\end{left}
Replacement:
\begin{left}
\begin{spacing}{1.5}
$y=y_0+U=e^x+U$

$y'=U'+e^x$

$U'+e^x=e^{2x}+e^x-2e^{2x}-2Ue^x+e^{2x}+U^2+2Ue^x$

\fbox{
$U'=U^2-$ Bernoulli equation
}
\end{spacing}
\end{left}
Replacement for Bernoulli's equation:
\begin{left}
\begin{spacing}{1.8}
$m=2; $ $\displaystyle \upsilon=U^{1-m}=\frac{1}{U}$

$\displaystyle U=\frac{1}{\upsilon}$

$\displaystyle U'=-\frac{\upsilon'}{\upsilon^2}$

$\displaystyle -\frac{\upsilon'}{\upsilon^2}=\frac{1}{\upsilon^2}$

$-\upsilon'=1 \Rightarrow \upsilon'=-1 \Rightarrow \upsilon=-x+C$

$\displaystyle U=\frac{1}{-x+C}$

\fbox{
$\displaystyle y=e^x-\frac{1}{x-C}=e^x-\frac{1}{x+\widetilde{C}}$
}
\end{spacing}
\end{left}

\begin{flushleft}
{\bf 179 } $ay'+ay=f(x); $ $a=const>0; $ $\lim\limits_{x \to 0} f(x) = b$

\end{flushleft}
Show that only one solution remains bounded at $x \rightarrow 0$ and find the limit of this solution at $x \rightarrow 0$

\begin{left}
\begin{spacing}{1.5}
$ay'+ay=f(x)$

$p(x)y+q(x)+r(x)y'=0$

$r(x)=x$

$p(x)=a$

$q(x)=f(x)$

\end{spacing}
\end{left}
Common decision:
\begin{left}
\begin{spacing}{1.8}
$\displaystyle y= e^{-\int\limits _{x_0}^x \frac{p(\tau)}{r(\tau)}\,d\tau} (C+\int\limits _{x_0}^x e^{\int\limits _{t_0}^t\frac{p(\tau)}{r(\tau)}\,d\tau}\frac{q(t)}{r(t)}\,dt)$

$\displaystyle y= e^{-\int\limits _{x_0}^x \frac{a}{\tau}\,d\tau} (C+\int\limits _{x_0}^x e^{\int\limits _{t_0}^t\frac{a}{\tau}\,d\tau}\frac{f(t)}{t}\,dt)$
\end{spacing}
\end{left}
$x_0=0$
\begin{left}
\begin{spacing}{3}
$\displaystyle y= e^{-aln(x)} (C+\int\limits _0^x \frac{1}{t^{-a}}\frac{f(t)}{t}\,dt)$

$\displaystyle y= \frac{1}{x^a} (C+\int\limits _0^x f(t)t^{a-1}\,dt)$

$\displaystyle \lim\limits_{x \to 0} \frac{C}{x^a} = \infty \Rightarrow C=0$

$\displaystyle y= \frac{1}{x^a}\int\limits _0^x f(t)t^{a-1}\,dt$

$\displaystyle \lim\limits_{x \to 0}y= \lim\limits_{x \to 0}\frac{1}{x^a}\int\limits _0^x f(t)t^{a-1}\,dt$

$\displaystyle \lim\limits_{x \to 0}y= \lim\limits_{x \to 0}\frac{1}{x^a}\int\limits _0^x bt^{a-1}\,dt$

$\displaystyle \lim\limits_{x \to 0}y= \lim\limits_{x \to 0}\frac{1}{x^a}\frac{b}{a}x^a$

\fbox{
$\displaystyle \lim\limits_{x \to 0}y= \frac{b}{a}$
}
\end{spacing}
\end{left}
\begin{flushleft}
{\bf 180 } $ay'+ay=f(x); $ $a=const<0; $ $\lim\limits_{x \to 0} f(x) = b$

\end{flushleft}
Show that all solutions of this equation have the same finite limit at $x \rightarrow 0$ . Find him.

From 179 follows the general solution of the equation:
\begin{left}
\begin{spacing}{3}
$\displaystyle y= \frac{1}{x^a} (C+\int\limits _0^x f(t)t^{a-1}\,dt)$

$\displaystyle y= \frac{b}{a}+\frac{1}{x^a} (C+\int\limits _0^x \epsilon(t)t^{a-1}\,dt)$

$\displaystyle \int\limits _0^x \epsilon(t)t^{a-1}\,dt$  limited $\forall C \Rightarrow$ \fbox{
$\lim\limits_{x \to 0}y= \frac{b}{a}$
}

$\displaystyle \int\limits _0^x \epsilon(t)t^{a-1}\,dt$  is not limited $\Rightarrow$

$\displaystyle \lim\limits_{x \to 0}\frac{C+\int\limits _0^x \epsilon(t)t^{a-1}\,dt}{x^a}=0+\frac{\int\limits _0^x \epsilon(t)t^{a-1}\,dt}{x^a}$

$\displaystyle \lim\limits_{x \to 0}\frac{\epsilon(x)x^a}{ax^a}=0$

\fbox{
$\lim\limits_{x \to 0}y= \frac{b}{a}$ $ \forall C$
}
\end{spacing}
\end{left}

\begin{flushleft}
{\bf 181 } $\displaystyle \frac{dx}{dt}+x=f(t); $ $f(t)$- continuous function $|f(f)|\leq M$ at$ -\infty<t<+\infty$

\end{flushleft}
Show that the equation has one bounded solution. Find this solution. Show that the found solution is periodic if the function $f(x)$ is periodic.
\begin{left}
\begin{spacing}{2}
$x'+x=f(t)-$ first-order linear differential equation with constant coefficients of the form: $x'-\lambda x=f(t)$

$\lambda=-1$
\end{spacing}
\end{left}
Common decision:
\begin{left}
\begin{spacing}{2}
$\displaystyle x(t)=e^{\lambda t}(C+\int\limits _s^t f(\tau)e^{-\lambda\tau}\,d\tau)$

$\displaystyle x(t)=e^{-t}C+e^{-t}\int\limits _{-\infty}^t f(\tau)e^{\tau}\,d\tau$

$\displaystyle |\int\limits _{-\infty}^t f(\tau)e^{\tau}\,d\tau| \leq Me^t$ 

$\displaystyle e^{-t}\int\limits _{-\infty}^t f(\tau)e^{\tau}\,d\tau \leq M \Rightarrow C=0$ $ \forall t -\infty<t<+\infty$ 

\fbox{
$\displaystyle x(t)=e^{-t}\int\limits _{-\infty}^t f(\tau)e^{\tau}\,d\tau$
}
\end{spacing}
\end{left}
\begin{flushleft}
Periodicity:

T-period
\end{flushleft}
\begin{left}
\begin{spacing}{1}
$\displaystyle x(t)=e^{-t}\int\limits _{-\infty}^t f(\tau+T)e^{\tau}\,d\tau$
\end{spacing}
\end{left}
\begin{left}
\begin{spacing}{1.5}
$\left[
  \begin{array}{ccc}
     \tau + T = z \\
     d\tau = dz \\
  \end{array}
\right]$
\end{spacing}
\end{left}

\begin{left}
\begin{spacing}{3}
$\displaystyle x(t)=e^{-t}\int\limits _{-\infty}^{t+T} f(z)e^{z-T}\,dz$

$\displaystyle x(t-T)=e^{-t}\int\limits _{-\infty}^{t+T} f(z)e^{z}\,dz = x(t+T)$
\end{spacing}
\end{left}


