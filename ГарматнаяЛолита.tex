\documentclass{article}

\usepackage{hyperref}
\usepackage[warn]{mathtext}
\usepackage[T2A]{fontenc}
\usepackage[utf8]{inputenc}
\usepackage[russian,english]{babel}
\title{Дифференциальные уравнения}
\author{Гарматная Лолита}
\date{Май 2022}

\begin{document}

\maketitle

\section*{Вариант 1}
\subsection*{№1}
\subsubsection*{a)  \begin{math} \left(2\,x\,y^{2}+3\,x^{2}+\frac{1}{x^{2}}+3\,\frac{x^{2}}{y^{2}}\right)\,\mathrm{d}x+\left(2\,x^{2}\,y+3\,y^{2}+\frac{1}{y^{2}}-2\,\frac{x^{3}}{y^{3}}\right)\,\mathrm{d}y=0
\newline P_y' = 4x\,y-\frac{6x^{2}}{y^{3}}
\newline Q_x' = 4x\,y-\frac{6x^{2}}{y^{3}}
\newline P_y'= Q_x' \Rightarrow
 \,Уравнение\quadв\quadполных\quadдифференциалах. 
\newline Для\quadрешения\quadиспользовать\quadформулу:
\newline u\left(x, y\right) = \int_{x_0}^{x}P\left(x, y_0\right)\, d{x}\ + \int_{y_0}^{y}Q\left(x, y\right)\,d{y}\
    \end{math}
}
\subsubsection*{b)  \begin{math} 
x\,y\,\left(1+y^{2}\right)\,\mathrm{d}x+\left(1+x^{2}\right)\,\mathrm{d}y=0
\newline Уравнение\quadс\quadразделяющимися\quadпеременными.
\newline С \quad помощью\quad интегрирующего\quad множителя\quad \mu\left(x, y\right)=\left(y(1+x^{2})\right)^{-1} \quad приводим \quad уравнение\newline к\quad уравнению \quad в \quad полных\quad дифференциалах \quadи \quadиспользуем \quadформулу \quadиз \quadпункта \quad a)\newline + \quad решаем \quad уравнение \quad
y(1+x^{2}) = 0. 
    \end{math}
}
\subsubsection*{c)  \begin{math} 2\,x\,y\,\mathrm{d}x-\left(x^{2}-y^{2}\right)\,\mathrm{d}y=0
\newline Уравнение\quad в\quad нормальной\quad дифференциальной \quad форме.
\newline Ищем\quad интегрирующий\quad множитель\quad в\quad виде\quad w\,=\,y^{2} +x^{2}:
\newline \psi\left(w\right)=\frac{4x}{-(x^{2}-y^{2})2\,x\,-2y\,2x\,y} = \frac{-2}{x^{2}+y^{2}}.
\newline \mu\left(w\right)=e^{-\int_{1}^{w} \frac{2}{\tau}\,d{\tau}} \newline  С\quad помощью \quadполученного\quad интегрирующего\quad множителя\quad  \mu\left(x^2+y^2\right)=\frac{1}{\left(x^2+y^2\right)^{2}}\quad приводим \quadуравнение\quad к\quad уравнению\quad в\quad полных\quad дифференциалах\quad и\quad решаем\newline его,\quad используя\quad формулу\quad из \quadпункта\quad a).
    \end{math}
}
\subsubsection*{d)  \begin{math} y'\,\ctg{x}-y=2\,\cos^{2}\left(x\right)\,\ctg{x}
\newline Линейное \quad дифференциальное \quad уравнение \quad первого\quad порядка.
\newline Для \quad решения \quad используют \quad метод \quad вариации \quad произвольной \quad постоянной.\newline Также\quad существует \quad метод \quad основанный \quad на \quad представлении \quad искомой \quad функции \quad y \newline в \quadвиде \quad y\left(x\right)=u\left(x\right)v\left(x\right).
    \end{math}
}
\subsubsection*{e)  \begin{math}
x^{2}\,y^{2}\,y'+x\,y^{3}=a^{2},\quad a\in R
\newline y'\,+\,\frac{1}{x}y\,=\,\frac{a^{2}}{x^{2}}y^{-2}
\newline Дифференциальное\quad уравнение\quad Бернулли.
\newline Дифференциальное \quadуравнение \quadБернулли\quad сводится\quad к \quadлинейному \quadдифференциальному \newline уравнению \quad подстановкой\quad z=y^{1-(-2)} \quad.
\end{math}
}
\subsubsection*{f)  \begin{math} 
y'=4\,y^{2}-4\,x^{2}\,y+x^{4}+x+4
\newline Уравнение \quad Риккати.
\newlineПусть \quad y_1\left(x\right)\, - \, частное \quadрешение. \quad Подстановкой \quad y \, = y_1 + u\quad уравнение \quadРиккати\newline приводится\quad к\quad уравнению \quadБернулли.
    \end{math}
}
\subsubsection*{g)  \begin{math}
y\,x'-2\,x+y^{2}=0
\newline Линейное \quad дифференциальное \quad уравнение \quad первого \quad порядка.
\newline Для \quad решения \quad используют \quad метод \quad вариации \quad произвольной \quad постоянной.
    \end{math}
}
\newpage
\subsection*{№2}
 \begin{math}
\left(x^{3}\,\left(1+\ln\left(x\right)\right)+2\,y\right)\,\mathrm{d}x+x\,\left(3\,y^{2}\,x^{2}-1\right)\,\mathrm{d}y=0;\quad  \mu=\mu\left(x\right) \, , \mu=\mu\left(y\right) \, .
\newline \psi\left(x\right)=\frac{2-9y^{2}\,x^{2} + 1}{x\left(3y^{2}x^{2}-1\right)}=\frac{-3}{x}
\newline  \psi\left(y\right)\neq\frac{2-9y^{2}\,x^{2} + 1}{-(x^{3}\left(1+\ln\left(x\right)\right)+2\,y)}
\newline Интегрирующий\quad множитель\quad будет\quad иметь\quad вид:
\newline \mu\left(x\right)=e^{\int_{1}^{x}\frac{3}{\tau}d{\tau}}=\frac{1}{x^{3}}
\newline Домножая\quad уравнение \quadна\quad интегрирующий\quad множитель,\quad получим\quad уравнение\newline в \quadполных\quad дифференциалах:
\newline \left(1+\ln\left(x\right) + 2\frac{y}{x^{3}}\right)\,\mathrm{d}x=0 +\left(3y^{2}-\frac{1}{x^2}\right)\,\mathrm{d}y=0
\newline P_y' = \frac{2}{x^{3}}
\newline Q_x' = \frac{2}{x^{3}}
\newline P_y'= Q_x'
\newline u\left(x, y\right) = \int_{1}^{x}(1+\ln\left(x\right))\, d{x}\ + \int_{0}^{y}(3y^{2}-\frac{1}{x^{2}})\,d{y}=
\newline = \left[\int_{1}^{x}\ln\left(x\right)=\left[u=\ln\left(x\right),\,\mathrm{d}u = \frac{1}{x}\mathrm{d}x,\, \mathrm{d}v=\mathrm{d}x, \, v=x\right]=  x\ln\left(x\right)-x+1\right]=
\newline = x\ln\left(x\right)+y^{3}-\frac{y}{x^{2}}
\newline Ответ: \quad y^{3}+x\ln\left(x\right)=c+\frac{y}{x^{2}}
    \end{math}
\subsection*{№3}
 \begin{math}
y\,\left(x+\ln\left(y\right)\right)+\left(x-\ln\left(y\right)\right)\,y'=0
\newline Замена: \quad u = \ln\left(y\right), \quad y=e^{u}, \quad u' = \frac{y'}{y}, \quad y' = u'\,e^{u}.
\newline e^{u}(x+u)+e^{u}u'(x-u)=0,
\newline (x+u)+u'(x-u)=0,
\newline u' = \frac{\mathrm{d}u}{\mathrm{d}x},
\newline (x-u)\mathrm{d}u=-(x+u)\mathrm{d}x,
\newline Замена: u=v\,x,\quad \mathrm{d}u=v\mathrm{d}x+s\mathrm{d}v,
\newline (-v^{2}\,x+2v\,x+x)\mathrm{d}x+(x^2-v\,x^{2})\mathrm{d}v=0
\newline P_v' =-2v\,x+2x, 
\newline Q_x' = 2x - 2v\,x,
\newline P_v'= Q_x' \Rightarrow \quad уравнение\quad в\quad
полных\quad дифференциалах.\newline u\left(x, y\right) = \int_{0}^{x}x\, d{x}\ + \int_{0}^{v}(x^{2}-v\,x^{2})\,d{v} = \frac{x^{2}}{2}+x^{2}v-\frac{v^2x^2}{2}
\newline v\,x^{2}+\frac{x^{2}}{2}-\frac{v^{2}\,x^{2}}{2}=c.
\newline К\quad замене \quad v=\frac{u}{x}:
\newline u\,x+\frac{x^{2}}{2}-\frac{u^{2}}{2}=c.
\newline К  \quad замене \quad u=\ln\left(y\right):
\newline x\ln\left(y\right)+\frac{x^{2}}{2}-\frac{\ln^{2}\left(y\right)}{2}=c.
\newline Ответ: \quad x\ln\left(y\right)-\frac{\ln^{2}\left(y\right)}{2}=c-\frac{x^{2}}{2}. 
\end{math}
\subsection*{№4}
 \begin{math}
\mathrm{d}y=\left(y-2\right)^{\frac{2}{3}}\,\mathrm{d}x,\quad y\left(1\right) = 2
\newline \frac{\mathrm{d}y}{(y-2)^{\frac{2}{3}}}=\mathrm{d}x,
\newline 3(y-2)^{\frac{1}{3}}=x+c,
\newline y = \left(\frac{x+c}{3}\right)^{3}+2.
\newline Используя \quad начальные\quad условия, \quadнайду\quad с:
\newline 2 = \left(\frac{1+c}{3}\right)^{3}+2,
\newline c = -1.
\newline y = \left(\frac{x-1}{3}\right)^{3}+2,
\newline y = \frac{x^{3}}{27}-\frac{x^{2}}{9}+\frac{x}{9}+\frac{53}{27}.
\newline Ответ: \quad y = \frac{x^{3}}{27}-\frac{x^{2}}{9}+\frac{x}{9}+\frac{53}{27}.
    \end{math}
\section*{Вариант 2}
\subsection*{№1}
\subsubsection*{a)  \begin{math} x\,y\,\left(1+y^{2}\right)\,\mathrm{d}x-\left(1+x^{2}\right)\,\mathrm{d}y=0
\newline Уравнение\quadс\quadразделяющимися\quadпеременными.
\newline С \quad помощью\quad интегрирующего\quad множителя\quad
\mu\left(x, y\right)=\left(y(1+y^{2})(1+x^{2})\right)^{-1} \quad приводим \newline уравнение\quad к\quad уравнению \quad в \quad полных\quad дифференциалах \quadи \quadиспользуя \quadформулу
\newline u\left(x, y\right) = \int_{x_0}^{x}P\left(x, y_0\right)\, d{x}\ + \int_{y_0}^{y}Q\left(x, y\right)\,d{y}\,, \quad решаем \quadполученное \quadуравнение.
\newlineРешаем \quad уравнение \quad
y(1+y^{2})(1+x^{2}) = 0. 
    \end{math}
}
\subsubsection*{b)  \begin{math} 
\frac{x^{2}\,\mathrm{d}y-y^{2}\,\mathrm{d}x}{\left(x-y\right)^{2}}=0
\newline P_y' = -\frac{2y(x-y)^{2}+2(x-y)y^{2}}{(x-y)^{4}}=\frac{-2y\,x}{(x-y)^{3}},
\newline Q_x' = \frac{2x(x-y)^{2}-2(x-y)x^{2}}{(x-y)^{4}}=\frac{-2y\,x}{(x-y)^{3}},
\newline P_y'= Q_x' \Rightarrow
 \,Уравнение\quadв\quadполных\quadдифференциалах. 
\newline Для\quadрешения\quadиспользовать\quadформулу:
\newline u\left(x, y\right) = \int_{x_0}^{x}P\left(x, y_0\right)\, d{x}\ + \int_{y_0}^{y}Q\left(x, y\right)\,d{y}\, .
    \end{math}
}
\subsubsection*{c)  \begin{math}
\left(x^{2}+y^{2}\right)\,\mathrm{d}x-2\,x\,y\,\mathrm{d}y=0
\newline Уравнение\quad в\quad нормальной\quad дифференциальной \quad форме.
\newline Ищем\quad интегрирующий\quad множитель\quad в\quad виде\quad w\,=\,x^{2} - y^{2}:
\newline \psi\left(w\right)=\frac{4y}{-2x\,y(2x)-(x^{2}+y^{2})(-2y)} = \frac{-2}{x^{2}-y^{2}}.
\newline \mu\left(w\right)=e^{-\int_{1}^{w} \frac{2}{\tau}\,d{\tau}} \newline  С\quad помощью \quadполученного\quad интегрирующего\quad множителя\quad  \mu\left(x^2-y^2\right)=\frac{1}{\left(x^2-y^2\right)^{2}}\quad приводим \quadуравнение\quad к\quad уравнению\quad в\quad полных\quad дифференциалах\quad и\quad решаем\newline его,\quad используя\quad формулу:
\newline u\left(x, y\right) = \int_{x_0}^{x}P\left(x, y_0\right)\, d{x}\ + \int_{y_0}^{y}Q\left(x, y\right)\,d{y}\, .
\newline Решить\quad уравнение\quad x^{2}-y^{2}=0.
    \end{math}
}
\subsubsection*{d)  \begin{math} 
\left(4-x^{2}\right)\,y'+x\,y=4
\newline y'+\frac{x}{4-x^{2}}\,y=\frac{4}{4-x^{2}} -\quad линейное \quad дифференциальное \quadуравнение\quad первого\quad порядка.
\newline Для \quadрешения\quad использовать \quad метод\quad вариации\quad произвольной\quad постоянной.
    \end{math}
}
\newpage
\subsubsection*{e)  \begin{math}
y'\,\tg\left(x\right)+2\,y\,\tg^{2}\left(x\right)=a\,y^{2},\quad  a\in R
\newline y'+2y\tg\left(x\right)=\frac{a}{\tg\left(x\right)}y^{2} - \quad дифференциальное \quad уравнение \quad Бернулли.
\newline Дифференциальное \quadуравнение \quadБернулли\quad сводится\quad к \quadлинейному \quadдифференциальному \newline уравнению \quad подстановкой\quad z=y^{1-2} \quad.
    \end{math}
}
\subsubsection*{f)  \begin{math} 
x\,y'=x^{2}\,y^{2}-y+1
\newline y'=x\,y^{2}-\frac{y}{x} +\frac{1}{x}-\quad уравнение \quad Риккати.
\newlineПусть \quad y_1\left(x\right)\, - \, частное \quadрешение. \quad Подстановкой \quad y \, = y_1 + u\quad уравнение \quadРиккати\newline приводится\quad к\quad уравнению \quadБернулли.
    \end{math}
}
\subsubsection*{g)  \begin{math}
\mathrm{d}x+\left(x+y^{2}\right)\,\mathrm{d}y=0
\newline Уравнение\quad в\quad нормальной\quad дифференциальной \quad форме.
\newline Ищем\quad интегрирующий\quad множитель\quad в\quad виде\quad w\,=\,y:
\newline \psi\left(y\right)=\frac{-1}{-1} = 1.
\newline \mu\left(y\right)=e^{\int_{1}^{y} \mathrm{d}\tau} \newline  С\quad помощью \quadполученного\quad интегрирующего\quad множителя\quad  \mu\left(y\right)=e^{y}\quad приводим \quadуравнение\newline к\quad уравнению\quad в\quad полных\quad дифференциалах\quad и\quad решаем\newline его,\quad используя\quad формулу:
\newline u\left(x, y\right) = \int_{x_0}^{x}P\left(x, y_0\right)\, d{x}\ + \int_{y_0}^{y}Q\left(x, y\right)\,d{y}\, .
    \end{math}
}
\subsection*{№2}
 \begin{math}
y^{2}\,\left(x-y\right)\,\mathrm{d}x+\left(1-x\,y^{2}\right)\,\mathrm{d}y=0;\quad  \mu=\mu\left(x\right), \mu=\mu\left(y\right) \, .
\newline \psi\left(x\right)\neq\frac{2y\,x-2y^{2}}{1-x\,y^{2}},
\newline \psi\left(y\right) = \frac{2y\,x-2y^{2}}{-y^{2}(x-y)}=\frac{2y(x-y)}{-y^{2}(x-y)}=-\frac{2}{y}, 
\newline \mu=\mu\left(y\right)=e^{-\int_{1}^{y}\frac{2}{\tau}\mathrm{d}\tau}=\frac{1}{y^{2}},
\newline Домножая\quad уравнение\quad на\quad интегрирующий\quad множитель,\quad получим\quad уравнение\quad в\newline полных\quad дифференциалах:
\newline (x-y)\mathrm{d}x+(\frac{1}{y^{2}}-x)\mathrm{d}y=0
\newline u\left(x, y\right) = \int_{0}^{x}x\, d{x}\ + \int_{0}^{y}(\frac{1}{y^{2}}-x)\,d{y}\, = \frac{x^{2}}{2}-\frac{1}{y}-x\,y .
\newline Ответ: \quad -x\,y-\frac{1}{y}=c-\frac{x^{2}}{2}, y = 0.
    \end{math}
\subsection*{№3}
\begin{math}
y'=x+e^{x+2y}
\newline Замена:\quad u=e^{-2y}, \quad y=-\frac{\ln\left(u\right)}{2}, \quad
y' = -\frac{u'}{2u}, \quad u'=-2e^{-2y}.
\newline Получается:
\newline -\frac{u'}{2u}=x+\frac{e^{x}}{u},
\newline u'+2u\,x=-2e^{x}
\newline Линейное \quad однородное \quad дифференциальное \quad уравнение.
\newline Для \quad решения \quad используется \quad метод \quad вариации \quad произвольной \quad постоянной.
\end{math}
\subsection*{№4}
 \begin{math}
\mathrm{d}y=x\,\sqrt{y}\,\mathrm{d}x,\quad y\left(1\right) = 0.
\newline \frac{\mathrm{d}y}{\sqrt{y}}=x\mathrm{d}x,
\newline 2y^{\frac{1}{2}}=\frac{x^{2}}{2}+c,
\newline y = \left(\frac{x^{2}}{4}+c\right)^{2},
\newline При \quadподстановке\quad начальных\quad данных:
\newline (\frac{1}{4}+c)^{2}=0 \quad \Rightarrow \quad c=-\frac{1}{4}.
\newline Ответ:\quad y=\left(\frac{x^{2}}{4}-\frac{1}{4}\right)^{2}.
    \end{math}


\end{document}
