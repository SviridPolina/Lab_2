\documentclass{article}
\usepackage{amsmath}
\usepackage[russian]{babel}
\usepackage[utf8]{inputenc}
\usepackage[letterpaper,top=2cm,bottom=2cm,left=3cm,right=3cm,marginparwidth=1.75cm]{geometry}

\title{Дифференциальные уравненя}
\author{Варвара Бородина}
\date{Март 2022}

\begin{document}

\maketitle

\section*{№169} 
$$xy^\prime - (2x + 1)y + y^2 = -x^2$$
Частное решение будет имеет вид: 
$$y_1(x) = ax + b$$
Подставим его в исходное уравнение:
$$ax - (2x + 1)(ax + b) + (ax + b)^2 = -x^2$$
$$ax - 2ax^2 - abx - ax - b + a^2x^2 + 2axb + b^2 = -x^2$$
Получаем систему:
$$ \begin{cases}
   a^2 - 2a = -1\\
   2ab - 2b = 0\\
   b^2 - b = 0\\
 \end{cases}$$
Система имеет два решения:
$$ \begin{cases}
   a = 1, b = 0 \\
   a = 1, b = 1
 \end{cases}$$
Пусть $a = 1, b = 0$. Тогда $y_1(x) = x$ - частное решение. Сделаем замену:
$$y = x + \frac{1}{z} \eqno(1)$$
Подставим замену в исходное уравнение:
$$x\left(1 - \frac{z^\prime}{z^2} \right) - (2x + 1)\left(x + \frac{1}{z}\right) + \left(x + \frac{1}{z}\right)^2 = -x^2$$
$$x - \frac{xz^\prime}{z^2} - 2x^2 - \frac{2x}{z} - x - \frac{1}{z} + x^2 + \frac{2x}{z} + \frac{1}{z^2} = -x^2$$
Получаем уравнение:
$$xz' + z - 1 = 0$$
Решаем уравнение методом Лагранжа:
$$xz' + z = 0$$
$$z' + \frac{z}{x} = 0$$
$$\frac{dz}{z} + \frac{dx}{x} = 0$$
$$\ln{z} + \ln{x} = \ln{c}$$
$$zx = c, z = \frac{c}{x}$$
$$z' = \frac{c'x - c}{x^2}$$
$$\frac{c'x - c}{x^2} + \frac{c}{x^2} = \frac{1}{x}$$
$$\frac{c'}{x} = \frac{1}{x}$$
$$c = x + c_1$$
$$ z = \frac{x + c_1}{x}$$
Подставим значение $z$ в уравнение (1):
$$y = x + \frac{x}{x + c_1}$$
Ответ: $y = x + \frac{x}{x + c_1}$
\section*{№170} 
$$y' - 2xy + y^2 = 5 - x^2$$
Частное решение будет имеет вид: 
$$y_1(x) = ax + b$$
Подставим его в исходное уравнение:
$$a - 2x(ax + b) + a^2x^2 + 2abx + b^2 = 5 - x^2$$
$$a - 2ax^2 - 2xb + a^2x^2 + 2abx + b^2 - 5 + x^2 = 0 $$
Получим систему уравнений:
$$ \begin{cases}
   -2a + a^2 + 1 = 0\\
   -2b + 2ab = 0\\
   a + b^2 - 5 = 0\\
 \end{cases}$$
Система имеет два решения:
 $$ \begin{cases}
   a = 1, b = 2\\
   a = 1, b = -2\\
 \end{cases}$$
Пусть $a = 1, b = 2$, тогда $y_1(x) = x + 2$ - частное решение.
Сделаем замену:
$$y = x + 2 + \frac{1}{z} \eqno(2)$$
Подставим замену в исходное уравнение:
$$1 - \frac{z'}{z^2} - 2x\left ( x + 2 + \frac{1}{z} \right ) +  x^2+  4 + \frac{4}{z} + \frac{1}{z^2} + 2x\left(2 + \frac{1}{z} \right) = 5 - x^2$$
$$1 - \frac{z'}{z^2} - 2x^2 - 4x - \frac{2x}{z} + x^2 + 4 + \frac{4}{z} + \frac{1}{z^2} + 4x + \frac{2x}{z} - 5 + x^2 = 0$$
Получаем уравнение:
$$z' - 4z - 1 = 0$$
Решаем уравнение методом Лагранжа:
$$z' - 4z = 0$$
$$\frac{dz}{z} - 4dx = 0$$
$$\ln{z} - 4x = c$$
$$z = e^{4x}c$$
$$z' = 4e^{4x}c + e^{4x}c'$$
$$4e^{4x}c + e^{4x}c' - 4e^{4x}c - 1 = 0$$
$$e^{4x}c' = 1$$
$$c' = e^{-4x}$$
$$c = -\frac{1}{4}e^{-4x} + c_1$$
$$z = -\frac{1}{4} + e^{4x}c_1$$
Подставим значение $z$ в уравнение (2):
$$y = x + 2 + \frac{1}{-\frac{1}{4} + e^{4x}c_1} = x + 2 + \frac{4}{e^{4x}c - 1}$$
Ответ: $y = x + 2 + \frac{4}{e^{4x}c - 1},  y = 2 + x$
\section*{№ 171}
$$y’+2ye^x-y^2=e^x(e^x+1)$$
Частное решение имеет вид:
$$y_1(x)=e^x+b$$
Подставим его в исходное уравнение:
$$e^x+2e^{2x}+2be^x-e^{2x}-2be^x-b^2=e^x(e^x+1)$$
$$-b^2=0$$
$$b=0$$
Частное решение $y_1(x)$ имеет вид:
$$y_1(x)=e^x$$
Сделаем замену в исходном уравнении:
$$y=y_1(x)+z=e^x+z \eqno(3)$$
$$(e^x+z)'+2(e^x+z)e^x-(e^x+z)^2-=e^x(e^x+1)$$
$$e^x+z'+2e^{2x}+2ze^x-e^{2x}-2ze^x-z^2=e^{2x}+e^x$$
Получаем уравнение:
$$z'=z^2$$
Решаем его методом Лагранжа:
$$z'-z^2=0,$$
$$\frac{dz}{dx}-z^2=0,$$
$$\frac{dz}{z^2}-\frac{dx}{1}=0,$$
$$\frac{dz}{z^2}-\frac{dx}{1}=0,$$
$$-\frac{1}{z}-x=c,$$
$$z=-\frac{1}{z+c},$$
Подставим значение $z$ в уравнение (3):
$$y=e^x-\frac{1}{z+c}$$
Ответ: $y=e^x-\frac{1}{z+c}$, $y=e^x$.
\section*{№ 179}
Общее уравнение имеет вид:
$$y=\frac{c}{|x|^a} + \frac{1}{|x|^a} \int_0^x f(t)|t|^{a-1} d(|t|),$$ 
где $d(|t|) = sign(t)dt$, $t\neq0$, или
$$y=\frac{c}{|x|^a} + \frac{b}{a} +\frac{1}{|x|^a} \int_0^x \varepsilon(t)|t|^{a-1} d(|t|) \eqno(4),$$ 
где $\varepsilon(t)\rightarrow0$ при $t\rightarrow0$ в силу условия. Вследствие оценки:
$$\frac{1}{|x|^a}\left|\int_0^x \varepsilon(t)|t|^{a-1} d(|t|)\right|\leq\frac{1}{a}\sup_{0\leq t\leq x}|\varepsilon(t)|\rightarrow0, x\rightarrow0.$$
Из (4) следует, что $\lim_{x\to 0}y$ существует и ограничен только при $c=0$ и равен $\frac{b}{a}$. Искомое решение уравнения имеет вид:
$$y=\frac{1}{|x|^a} \int_0^x f(t)|t|^{a-1} d(|t|)$$
\section*{№ 180}
$$xy' + ay = f(x), a = const < 0, f \rightarrow b, x \rightarrow 0$$
Общее решение уравнения имеет вид:
$$y(x) = \frac{C}{|x|^a} + \frac{1}{|x|^a}\int f(x)|x|^{a-1}d(|x|)$$
$$y(x) = \frac{C}{|x|^a} + \frac{b}{a} + \frac{1}{|x|^a}\int \varepsilon(x)|x|^{a-1}d(|x|)$$
Пусть $\int \varepsilon(x)|x|^{a-1}d(|x|)$ - ограничен, тогда:
$$\forall \; C \; \lim_{x\to 0}  y(x) = \frac{b}{a}$$
Пусть $\int \varepsilon(x)|x|^{a-1}d(|x|)$ - не ограничен, тогда применим правило Лопиталя:
$$\lim_{x\to 0} \frac{\int \varepsilon(x)|x|^{a-1}d(|x|)}{|x|^a} = \lim_{x\to0} \frac{\epsilon(x)|x|^{a-1}}{a|x|^{a-1}} = 0$$
Таким образом,
$$\lim_{x\to 0} y(x) = \frac{b}{a}, \; \forall \; C$$
\section*{№ 181}
Представим общее решение заданного уравнения в виде:
$$x(t) = ce^{-t}+e^{-t}\int_{-\infty}^t f(\tau)e^{\tau}d\tau \eqno(5)$$
Такое представление возможно в силу того, что:
$$\left|\int_{-\infty}^t f(\tau)e^{\tau}d\tau \right| \leq Me^t\eqno(6), $$
И, как следствие, несобственный интеграл сходится. Из неравенства (6) также следует, что функция $e^{-t}\int_{-\infty}^t f(\tau)e^{\tau}d\tau$ ограничена числом M $ \forall t \in (-\infty,+\infty)$. Таким образом необходимым и достаточным условием ограниченности функции x является равенство $c=0$. Искомое решение имеет вид:
$$x(t)=e^{-t}\int_{-\infty}^t f(\tau)e^{\tau}d\tau \eqno(7)$$
Пусть, далее:
$$ \forall \tau \in (-\infty,+\infty) f(\tau +T)=f(\tau),\; T>0$$
Тогда из (7) находим:
$$x(t) = e^{-t} \int^t_{-\infty} f(\tau + T)e^\tau d\tau = e^{-(t+T)} \int^{t+T}_{-\infty} f({\tau}_1)e^{{\tau}_1}d{\tau}_1 = x(t+T),$$
где $\tau_1=\tau+T$. Следовательно, x - периодическая функция.
\end{document} \documentclass{article}
\usepackage{amsmath}
\usepackage[russian]{babel}
\usepackage[utf8]{inputenc}
\usepackage[letterpaper,top=2cm,bottom=2cm,left=3cm,right=3cm,marginparwidth=1.75cm]{geometry}

\title{Дифференциальные уравненя}
\author{Варвара Бородина}
\date{Март 2022}

\begin{document}

\maketitle

\section*{№169} 
$$xy^\prime - (2x + 1)y + y^2 = -x^2$$
Частное решение будет имеет вид: 
$$y_1(x) = ax + b$$
Подставим его в исходное уравнение:
$$ax - (2x + 1)(ax + b) + (ax + b)^2 = -x^2$$
$$ax - 2ax^2 - abx - ax - b + a^2x^2 + 2axb + b^2 = -x^2$$
Получаем систему:
$$ \begin{cases}
   a^2 - 2a = -1\\
   2ab - 2b = 0\\
   b^2 - b = 0\\
 \end{cases}$$
Система имеет два решения:
$$ \begin{cases}
   a = 1, b = 0 \\
   a = 1, b = 1
 \end{cases}$$
Пусть $a = 1, b = 0$. Тогда $y_1(x) = x$ - частное решение. Сделаем замену:
$$y = x + \frac{1}{z} \eqno(1)$$
Подставим замену в исходное уравнение:
$$x\left(1 - \frac{z^\prime}{z^2} \right) - (2x + 1)\left(x + \frac{1}{z}\right) + \left(x + \frac{1}{z}\right)^2 = -x^2$$
$$x - \frac{xz^\prime}{z^2} - 2x^2 - \frac{2x}{z} - x - \frac{1}{z} + x^2 + \frac{2x}{z} + \frac{1}{z^2} = -x^2$$
Получаем уравнение:
$$xz' + z - 1 = 0$$
Решаем уравнение методом Лагранжа:
$$xz' + z = 0$$
$$z' + \frac{z}{x} = 0$$
$$\frac{dz}{z} + \frac{dx}{x} = 0$$
$$\ln{z} + \ln{x} = \ln{c}$$
$$zx = c, z = \frac{c}{x}$$
$$z' = \frac{c'x - c}{x^2}$$
$$\frac{c'x - c}{x^2} + \frac{c}{x^2} = \frac{1}{x}$$
$$\frac{c'}{x} = \frac{1}{x}$$
$$c = x + c_1$$
$$ z = \frac{x + c_1}{x}$$
Подставим значение $z$ в уравнение (1):
$$y = x + \frac{x}{x + c_1}$$
Ответ: $y = x + \frac{x}{x + c_1}$
\section*{№170} 
$$y' - 2xy + y^2 = 5 - x^2$$
Частное решение будет имеет вид: 
$$y_1(x) = ax + b$$
Подставим его в исходное уравнение:
$$a - 2x(ax + b) + a^2x^2 + 2abx + b^2 = 5 - x^2$$
$$a - 2ax^2 - 2xb + a^2x^2 + 2abx + b^2 - 5 + x^2 = 0 $$
Получим систему уравнений:
$$ \begin{cases}
   -2a + a^2 + 1 = 0\\
   -2b + 2ab = 0\\
   a + b^2 - 5 = 0\\
 \end{cases}$$
Система имеет два решения:
 $$ \begin{cases}
   a = 1, b = 2\\
   a = 1, b = -2\\
 \end{cases}$$
Пусть $a = 1, b = 2$, тогда $y_1(x) = x + 2$ - частное решение.
Сделаем замену:
$$y = x + 2 + \frac{1}{z} \eqno(2)$$
Подставим замену в исходное уравнение:
$$1 - \frac{z'}{z^2} - 2x\left ( x + 2 + \frac{1}{z} \right ) +  x^2+  4 + \frac{4}{z} + \frac{1}{z^2} + 2x\left(2 + \frac{1}{z} \right) = 5 - x^2$$
$$1 - \frac{z'}{z^2} - 2x^2 - 4x - \frac{2x}{z} + x^2 + 4 + \frac{4}{z} + \frac{1}{z^2} + 4x + \frac{2x}{z} - 5 + x^2 = 0$$
Получаем уравнение:
$$z' - 4z - 1 = 0$$
Решаем уравнение методом Лагранжа:
$$z' - 4z = 0$$
$$\frac{dz}{z} - 4dx = 0$$
$$\ln{z} - 4x = c$$
$$z = e^{4x}c$$
$$z' = 4e^{4x}c + e^{4x}c'$$
$$4e^{4x}c + e^{4x}c' - 4e^{4x}c - 1 = 0$$
$$e^{4x}c' = 1$$
$$c' = e^{-4x}$$
$$c = -\frac{1}{4}e^{-4x} + c_1$$
$$z = -\frac{1}{4} + e^{4x}c_1$$
Подставим значение $z$ в уравнение (2):
$$y = x + 2 + \frac{1}{-\frac{1}{4} + e^{4x}c_1} = x + 2 + \frac{4}{e^{4x}c - 1}$$
Ответ: $y = x + 2 + \frac{4}{e^{4x}c - 1},  y = 2 + x$
\section*{№ 171}
$$y’+2ye^x-y^2=e^x(e^x+1)$$
Частное решение имеет вид:
$$y_1(x)=e^x+b$$
Подставим его в исходное уравнение:
$$e^x+2e^{2x}+2be^x-e^{2x}-2be^x-b^2=e^x(e^x+1)$$
$$-b^2=0$$
$$b=0$$
Частное решение $y_1(x)$ имеет вид:
$$y_1(x)=e^x$$
Сделаем замену в исходном уравнении:
$$y=y_1(x)+z=e^x+z \eqno(3)$$
$$(e^x+z)'+2(e^x+z)e^x-(e^x+z)^2-=e^x(e^x+1)$$
$$e^x+z'+2e^{2x}+2ze^x-e^{2x}-2ze^x-z^2=e^{2x}+e^x$$
Получаем уравнение:
$$z'=z^2$$
Решаем его методом Лагранжа:
$$z'-z^2=0,$$
$$\frac{dz}{dx}-z^2=0,$$
$$\frac{dz}{z^2}-\frac{dx}{1}=0,$$
$$\frac{dz}{z^2}-\frac{dx}{1}=0,$$
$$-\frac{1}{z}-x=c,$$
$$z=-\frac{1}{z+c},$$
Подставим значение $z$ в уравнение (3):
$$y=e^x-\frac{1}{z+c}$$
Ответ: $y=e^x-\frac{1}{z+c}$, $y=e^x$.
\section*{№ 179}
Общее уравнение имеет вид:
$$y=\frac{c}{|x|^a} + \frac{1}{|x|^a} \int_0^x f(t)|t|^{a-1} d(|t|),$$ 
где $d(|t|) = sign(t)dt$, $t\neq0$, или
$$y=\frac{c}{|x|^a} + \frac{b}{a} +\frac{1}{|x|^a} \int_0^x \varepsilon(t)|t|^{a-1} d(|t|) \eqno(4),$$ 
где $\varepsilon(t)\rightarrow0$ при $t\rightarrow0$ в силу условия. Вследствие оценки:
$$\frac{1}{|x|^a}\left|\int_0^x \varepsilon(t)|t|^{a-1} d(|t|)\right|\leq\frac{1}{a}\sup_{0\leq t\leq x}|\varepsilon(t)|\rightarrow0, x\rightarrow0.$$
Из (4) следует, что $\lim_{x\to 0}y$ существует и ограничен только при $c=0$ и равен $\frac{b}{a}$. Искомое решение уравнения имеет вид:
$$y=\frac{1}{|x|^a} \int_0^x f(t)|t|^{a-1} d(|t|)$$
\section*{№ 180}
$$xy' + ay = f(x), a = const < 0, f \rightarrow b, x \rightarrow 0$$
Общее решение уравнения имеет вид:
$$y(x) = \frac{C}{|x|^a} + \frac{1}{|x|^a}\int f(x)|x|^{a-1}d(|x|)$$
$$y(x) = \frac{C}{|x|^a} + \frac{b}{a} + \frac{1}{|x|^a}\int \varepsilon(x)|x|^{a-1}d(|x|)$$
Пусть $\int \varepsilon(x)|x|^{a-1}d(|x|)$ - ограничен, тогда:
$$\forall \; C \; \lim_{x\to 0}  y(x) = \frac{b}{a}$$
Пусть $\int \varepsilon(x)|x|^{a-1}d(|x|)$ - не ограничен, тогда применим правило Лопиталя:
$$\lim_{x\to 0} \frac{\int \varepsilon(x)|x|^{a-1}d(|x|)}{|x|^a} = \lim_{x\to0} \frac{\epsilon(x)|x|^{a-1}}{a|x|^{a-1}} = 0$$
Таким образом,
$$\lim_{x\to 0} y(x) = \frac{b}{a}, \; \forall \; C$$
\section*{№ 181}
Представим общее решение заданного уравнения в виде:
$$x(t) = ce^{-t}+e^{-t}\int_{-\infty}^t f(\tau)e^{\tau}d\tau \eqno(5)$$
Такое представление возможно в силу того, что:
$$\left|\int_{-\infty}^t f(\tau)e^{\tau}d\tau \right| \leq Me^t\eqno(6), $$
И, как следствие, несобственный интеграл сходится. Из неравенства (6) также следует, что функция $e^{-t}\int_{-\infty}^t f(\tau)e^{\tau}d\tau$ ограничена числом M $ \forall t \in (-\infty,+\infty)$. Таким образом необходимым и достаточным условием ограниченности функции x является равенство $c=0$. Искомое решение имеет вид:
$$x(t)=e^{-t}\int_{-\infty}^t f(\tau)e^{\tau}d\tau \eqno(7)$$
Пусть, далее:
$$ \forall \tau \in (-\infty,+\infty) f(\tau +T)=f(\tau),\; T>0$$
Тогда из (7) находим:
$$x(t) = e^{-t} \int^t_{-\infty} f(\tau + T)e^\tau d\tau = e^{-(t+T)} \int^{t+T}_{-\infty} f({\tau}_1)e^{{\tau}_1}d{\tau}_1 = x(t+T),$$
где $\tau_1=\tau+T$. Следовательно, x - периодическая функция.
\end{document} 