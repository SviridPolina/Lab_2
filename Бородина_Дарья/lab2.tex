\documentclass{article}

\usepackage[russian]{babel}
\usepackage{mathtext}
\usepackage{amsmath}

\title{Лабораторная работа 2}
\author{Бородина Дарья Евгеньевна}
\date{}


\begin{document}

\maketitle
\section*{Вариант 1}
\section*{№1}
\subsection*{a)}
\subsubsection*{Тип:} 
уранение в полных дифференциалах

\begin{equation*}
\frac{\partial P}{\partial y}=4xy-\frac{6x^2}{y^3}=
\frac{\partial Q}{\partial x}=4xy-\frac{6x^2}{y^3}
\end{equation*}

\subsubsection*{Метод:}
с помощью криволинейного интеграла второго рода

\begin{equation*}
\int_{x_0}^{x}\left(2xy^2+3x^2+\frac{1}{x^2}+\frac{3x^2}{y^2}\right)dx+\int_{y_0}^{y}\left(2x_0^2y+3y^2+\frac{1}{y^2}-\frac{2x_0^3}{y^3}\right)dy=u(x,y)
\end{equation*}

\subsection*{б)}
\subsubsection*{Тип:} 
уранение с разделяющимися переменными
\subsubsection*{Метод:} 
разделить переменные и проинтегрировать слева и справа

\begin{equation*}
\int \frac{xdx}{1+x^2}=-\int \frac{dy}{y(1+y^2)}
\end{equation*}

\subsection*{в)}
\subsubsection*{Тип:} 
однородное уранение

\subsubsection*{Метод:}
диффеомормная замена перемнных

\begin{gather*}
\begin{cases}
\frac{y}{x}=u \\
x=x
\end{cases}\\
y=ux, \quad   dy=xdu+udx  \\
    2x^2udx-(x^2-u^2x^2)(xdu+udx)=0 \\
    2x^2udx-x^3du-x^2udx+u^2x^3du+u^3x^2dx=0 \\
    (x^2u+u^3x^2)dx+(-x^3+u^2x^3)du=0 \quad |:x^2 \\
    x'(u+u^3)+x(-1+u^2)=0
\end{gather*}
получили линейное по x ураннение, которое решается методом Лагранжа

\subsection*{г)}
\subsubsection*{Тип:} 
линейное по y уравнение 

\subsubsection*{Метод:}
метод Лагранжна

\subsection*{д)}
\subsubsection*{Тип:} 
уравнение Бернулли по y, $m=-2$

\subsubsection*{Метод:}
замена переменных и сведение к линейному уравнению
\begin{gather*}
    u=y^{1-m}=y^3, \quad   du=3y^2dy
\end{gather*}

\subsection*{е)}
\subsubsection*{Тип:} 
уравнение Рикатти

\subsubsection*{Метод:}
не решается в общем случае

\subsection*{ж)}
\subsubsection*{Тип:} 
линейное по x уравнение 

\subsubsection*{Метод:}
метод Лагранжа

\section*{№2}

\begin{equation*}
    (x^3+x^3\ln{x}+2y)dx+(3y^2x^3-x)dy=0
\end{equation*}

\begin{equation*}
    \Psi(y): \quad \frac{Q'_x-P'_y}{P}=
    \frac{9x^2y^2-3}{x^3+x^3\ln(x)+2y}=f(x,y) \Rightarrow \Psi\neq\Psi(y)
\end{equation*}

\begin{equation*}
    \Psi(x): \quad \frac{P'_y-Q'_x}{Q}=
    \frac{3-9x^2y^2}{3x^3y^2-x}=
    \frac{3(1-3x^2y^2)}{-x(1-3x^2y^2)}=-\frac{3}{x}
    \Rightarrow \Psi=\Psi(x)
\end{equation*}

\begin{equation*}
    \mu(x)=e^{\int\Psi(x)dx}=e^{-\int\frac{3}{x}dx}=e^{-3\ln x}=x^{-3}
\end{equation*}

\begin{equation}
    \left(1+\ln(x)+\frac{2y}{x^3}\right)dx+\left(3y^2+-\frac{1}{x^2}\right)dy=0 \label{pdd},\quad x\neq0
\end{equation}

\begin{equation*}
    \frac{\partial P_1}{\partial y}=\frac{\partial Q_1}{\partial x}=\frac{2}{x^3}
\end{equation*}

(\ref{pdd}) - уравнение в полнных дифференциалах

\begin{equation*}
    \frac{\partial u}{\partial x} = 1+\ln x+\frac{2y}{x^3}
\end{equation*}

\begin{equation*}
    u(x,y)= \int \left(1+\ln x+\frac{2y}{x^3}\right)dx
\end{equation*}

\begin{equation*}
    u(x,y)= x+x\ln x-x-\frac{y}{x^2}+c(y)
\end{equation*}

\begin{equation*}
    \frac{\partial u}{\partial y} = -\frac{1}{x^2}+c'(y)
\end{equation*}

\begin{equation*}
    c'(y)=3y^2
\end{equation*}

\begin{equation*}
    c(y)=y^3
\end{equation*}

\begin{equation*}
    u(x,y)=x\ln x-\frac{y}{x^2}+y^3=c
\end{equation*}

Ответ: $x\ln x-\frac{y}{x^2}+y^3=c$.

\section*{№3}

\begin{equation*}
    y(x+\ln y)+(x-\ln y)y'=0 \quad \ln y = \eta, \quad d\eta=\frac{dy}{y}
\end{equation*}

\begin{equation*}
    \frac{dy}{d\eta}(x+\eta)+(x-\eta)\frac{dy}{dx}=0 \quad |:dy
\end{equation*}

\begin{equation*}
    (x+\eta)dx+(x-\eta)d\eta=0
\end{equation*}

\begin{equation*}
    \frac{\partial P}{\partial \eta}=\frac{\partial Q}{\partial x}=1 \Rightarrow \text{уравнение в полных дифференциалах}
\end{equation*}

Метод: с помощью криволинейного интеграла второго рода
\begin{equation*}
    \int_{x_0}^{x}(x+\eta)dx+\int_{\eta_0}^{\eta}(x-\eta)d\eta=u(x,y)
\end{equation*}

\begin{equation*}
    u(x,y)=\frac{x^2}{2}+x\eta-\frac{\eta^2}{2}=c
\end{equation*}

\begin{equation*}
    \frac{x^2}{2}+x\ln y-\frac{\ln^2{y}}{2}=c
\end{equation*}

Ответ:$\frac{x^2}{2}+x\ln y-\frac{\ln^2{y}}{2}=c$

\section*{№4}

\begin{equation*}
    dy=(y-2)^{\frac{2}{3}}dx,\quad y|_{x=1}=2
\end{equation*}

\begin{equation*}
    d(y-2)(y-2)^{-\frac{2}{3}}=dx, \quad y-2=t
\end{equation*}

\begin{equation*}
    \int t^{-\frac{2}{3}}dt=\int dx
\end{equation*}

\begin{equation*}
    3t^{\frac{1}{3}}=x+c
\end{equation*}

\begin{equation*}
    3(y-2)^{\frac{1}{3}}=x+c
\end{equation*}

\begin{equation*}
    0=1+c \Rightarrow c=-1
\end{equation*}

\begin{equation*}
    3(y-2)^{\frac{1}{3}}=x-1
\end{equation*}

Ответ: $x=1+3(y-2)^{\frac{1}{3}}$.

\section*{Вариант 2}
\section*{№1}
\subsection*{a)}
\subsubsection*{Тип:} 
уравнение с разделяющимися переменными
\subsubsection*{Метод:}
разделить переменные и проинтегрировать слева и справа

\begin{equation*}
\frac{x}{1 + x^2}dx - \frac{1}{y(1+y^2)}dy = 0
\end{equation*}

\subsection*{б)}
\subsubsection*{Тип:} 
однородное уравнение нулевой степени
\subsubsection*{Метод:}
диффеомормная замена перемнных

\begin{gather*}
\begin{cases}
\frac{y}{x}=u \\
x=x
\end{cases}\\
y=ux, \quad   dy=xdu+udx 
\end{gather*}

\subsection*{в)}
\subsubsection*{Тип:} 
однородное уравнение второй степени
\subsubsection*{Метод:}
диффеомормная замена перемнных

\begin{gather*}
\begin{cases}
\frac{y}{x}=u \\
x=x
\end{cases}\\
y=ux, \quad   dy=xdu+udx 
\end{gather*}

\subsection*{г)}
\subsubsection*{Тип:} 
линейное по уравнение y

\subsubsection*{Метод:}
метод Бернули

\subsection*{д)}
\subsubsection*{Тип:} 
уравнение Бернули

\subsubsection*{Метод:}
замена переменных и сведение к линейному уравнению

\begin{gather*}
    u=y^{1-m}
\end{gather*}

\subsection*{е)}
\subsubsection*{Тип:} 
уравнение Риккати

\subsubsection*{Метод:}
не решается в общем случае


\subsection*{ж)}
\subsubsection*{Тип:} 
линейное по x

\subsubsection*{Метод:}
метод Бернули 

\section*{№2}

\begin{equation*}
y^2(x - y)dx + (1 - xy^2)dy = 0
\end{equation*}

\begin{equation*}
\mu = \mu(y): \frac{Q'_x - P'_y}{P} = \frac{-y^2 - (2xy - 3y^2)}{y^2(x - y)}=\frac{-2y(x-y)}{y^{2}(x-y)} => \mu = \mu(y)
\end{equation*}


\begin{equation*}
    \mu' = \frac{2}{y}\mu
\end{equation*}

\begin{equation*}
    \frac{d\mu}{\mu} = - \frac{2dy}{y}
\end{equation*}

\begin{equation*}
    \ln{\mu} = \ln{y^{-2}}
\end{equation*}

\begin{equation*}
    \mu = y^{-2}
\end{equation*}

\begin{equation*}
    (x - y)dx + (\frac{1}{y^2}-x)dy = 0\\
\end{equation*}

\begin{equation*}
\frac{\partial P}{\partial y}=1=
\frac{\partial Q}{\partial x} \Rightarrow \text{ уравнение в полных дифференциалах}
\end{equation*}

\begin{equation*}
\int_{0}^{x}(x-y)dx+\int_{0}^{y}\left(\frac{1}{y^2}-x_0\right)dy=u(x,y)
\end{equation*}

Ответ: $ u(x,y)=x^2 - xy - \frac{2}{y}$

\section*{№3}

\begin{equation*}
    y' = x + e^{x + 2y}
\end{equation*}

\begin{equation*}
   \eta = e^{-2y}, \quad    d\eta = -2e^{-2y}dy
\end{equation*}

\begin{equation*}
    \frac{dy}{dx} = x + e^{x + 2y}
\end{equation*}

\begin{equation*}
    \frac{d\eta}{-2\eta dx} = x + \frac{e^{x}}{\eta}
\end{equation*}

\begin{equation*}
    e^{-2y}dy = (xe^{-2y} + e^x)dx
\end{equation*}

\begin{equation*}
    \eta '+2\eta x=e^x
\end{equation*}
линейное уравнение по $\eta$, для решения применяем метод Лагранжа


\section*{№4}

\begin{equation*}
    dy = x\sqrt{y} dx,y|_{x=1} = 0
\end{equation*}
\begin{equation*}
    xdx = \frac{dy}{\sqrt{y}}
\end{equation*}
\begin{equation*}
    \frac{x^2}{2} = 2\sqrt{y} + C
\end{equation*}

\begin{equation*}
    C = \frac{1}{2}
\end{equation*}

\begin{equation*}
    x^2 - 4\sqrt{y} = 1
\end{equation*}

Ответ: $x^2 - 4\sqrt{y} = 1$

\end{document}
